\documentclass{uofa-eng-assignment}
\usepackage[utf8]{inputenc}
\usepackage{geometry}
\geometry{
 a4paper,
 left=25.4mm,
 top=25.4mm,
 }
 \usepackage[spanish]{babel}
 \usepackage{graphicx} 
 \usepackage{float} 
 \usepackage{natbib}
 \usepackage{subcaption}
\usepackage{caption}
\usepackage{wrapfig}
 \usepackage{hyperref}
 \hypersetup{
    colorlinks=true,
    linkcolor=blue,
    filecolor=magenta,      
    urlcolor=cyan,
    pdftitle={Overleaf Example},
    pdfpagemode=FullScreen,
    }

\usepackage{bookmark}
\newcommand*{\name}{Natalia Opazo}

\newcommand*{\course}{2do Control MGR 622. “Evaluación de recursos acuáticos”  \\
Diplomado en Evaluación de Recursos Pesqueros}
%\newcommand*{\assignment}{Assignment 1}

\begin{document}

\maketitle

1. Identifique una pesquería/recurso de su interés y genere una tabla con los parámetros biológico/pesqueros relevantes. Si realiza cálculos intermedios y/o supuestos debe indicarlos/explicarlos/justificarlos brevemente (5 puntos). \\

Considerando un modelo edad-estructurado determine:\\

2. Los niveles de mortalidad por pesca límite F20\%B0 (agotamiento) y objetivo F40\%B0. Explique (10 puntos).\\


3. El nivel de reducción/agotamiento poblacional (\%B0) si la mortalidad por pesca para la pesquería se encuentra en F=2.5M.  ¿Cuál es el diagnóstico de la población? (10 puntos).\\


4. ¿En cuánto se debe reducir el esfuerzo de pesca para recuperar la población al objetivo de manejo?  ¿Cuál es el efecto en el rendimiento de pesca de largo plazo para la pesquería actual?  Comente (10 puntos)\\


5. La edad y longitud de selectividad (a/L50, A/L95) factible que permita recuperar la población al objetivo de manejo sin alterar el nivel del esfuerzo de pesca. Comente (10 puntos)\\


\section*{Anexos}


\end{document}
